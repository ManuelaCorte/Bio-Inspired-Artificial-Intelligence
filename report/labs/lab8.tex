\subsection{Exercise 1}
\subsubsection*{Or}
Using standard parameters
\subsubsection*{And}
Stuck at fitness 0.012 even changing parameters --> change in topology adding a hidden node
\subsubsection*{Xor}
Need to add nodes in the hidden layer since non linearly separable problem (2 hidden nodes.
We can write the Xor as a XOR b = (a AND NOT b)OR(b AND NOT a)
the two hidden nodes are the two possible configurations of the or

With standard parameters best solution has early convergence to bad overall fitness value (0.668191835386526) --> bigger / more frequent mutation to prevent best solution from getting stuck (0.8 mutation rate) 
\subsection{Exercise 2}
\subsubsection*{Temporal Or}
Using a recurrent architecture allows us to take into consideration local structure in the sequence of inputs (in our case a temporal sequence) while each execution of a standard FFNN is only concerned with its current inputs

\subsubsection*{Temporal And}
Same parameters as temporal or (maybe because the problem is easy enough that we don't need to be too precise in the es parameters?)

\subsubsection*{Temporal Xor}
\subsection{Exercise 3}
\subsubsection*{No elitism}
Doesn't converge in standard setting

\subsubsection*{Elitism}
Reaches fitness threshold after 36 generations
Elitism in speciation probably helps retaining very promising species (and thus well performing topologies) and also keeping the number of species that compete lower

Both cases starting with a more complex topology (hidden layer with one node gives worse results)
\subsection{Exercise 4}