\subsection{Exercise 1}
Boid characteristics:
\begin{itemize}
    \item separation: boid maintains given distance from other boids
    \item cohesion: boid moves towards center of mass of the neighbouring boids (in our case the entire swarm is considered the neighbourhood)
    \item alignment: boid aligns its angle along those of neighbouring boids 
\end{itemize}

In general if we have a big separation coefficient (spaced out population) then we need a low cohesion otherwise the boids in the population tends to jump around (conversely if we have higi cohesion then we need low separation and we have a swarm where boids are very close to each other). In general if we don't have obstacle a large alignment helps the movements resemble more those of a real flock

\subsection{Exercise 2}

\subsection{General Questions}
When do you think it is useful to have a lower (higher) cognitive learning rate? What about the social learning rate?

From a biological point of view, which neighborhood topology do you consider as the most plausible?
